\documentclass{article}
\usepackage{graphicx} % Required for inserting images
\usepackage{amsmath}
\usepackage{enumitem}
\usepackage{hyperref}
\usepackage{multicol}


\begin{document}

    \pagenumbering{roman}

    \tableofcontents
    \pagebreak

    \pagenumbering{arabic}

    \begin{center}
        \subsection*{Boolean Algebra}
        \addcontentsline{toc}{section}{Boolean Algebra}
    \end{center}

    \medskip
    \subsubsection*{Basics}
    \addcontentsline{toc}{subsection}{Basics}
    \vspace{-1em}
    \rule{\linewidth}{0.05mm}


    \renewcommand{\arraystretch}{1.6}
    \vspace{2em}
    \begin{center}
        \begin{tabular}{ r | l l l l}
            & $X+0=0$ & $X+1=1$ & $X\cdot0=0$ & $X\cdot 1 = X$ \\
            Basic & $X+X=X$ & $X\cdot X=X$ \\
            & $(X')'=X$ & $X+X'=1$ & $X\cdot X'=0$ \\
        \end{tabular}
    \end{center}

    \smallskip
    \begin{center}
        \begin{tabular}{ r | l l}
            DeMorgan's Laws & $(X+Y)'=X'Y'$ & $(XY)' = X'+Y'$ \\
        \end{tabular}
    \end{center}

    \medskip\noindent
    \subsubsection*{Algebraic Simplification}
    \addcontentsline{toc}{subsection}{Algebraic Simplification}
    \vspace{-1em}
    \rule{\linewidth}{.1mm}

    \smallskip\noindent

    \begin{enumerate}
        \item Combine terms using the uniting theorem.
        \item Absorb terms using the absorption theorem.
        \item Eliminate literals using the elimination theorem.
        \item Add redundant terms, if needed, to enable further simplification.
    \end{enumerate}

    \smallskip\noindent
    Use the sum-of-products (SOP) theorems to get a SOP result, and the product-of-sums (POS) theorems to get a POS result.

    \medskip
    \begin{center}
        \begin{tabular}{ r | l l}
            & SOP         & POS             \\
            \hline
            Uniting     & $XY+XY'=X$  & $(X+Y)(X+Y')=X$ \\
            Absorption  & $X+XY=X$    & $X(X+Y)=X$      \\
            Elimination & $X+X'Y=X+Y$ & $X(X'+Y)=XY$
        \end{tabular}
    \end{center}

    \pagebreak
    \smallskip\noindent
    \subsubsection*{Expanding and Factoring Boolean Expressions}
    \addcontentsline{toc}{subsection}{Expanding and Factoring Boolean Expressions}
    \vspace{-1em}
    \rule{\linewidth}{0.1mm}

    \vspace{.2em}\noindent
    Starting with a POS expression, the SOP expression can be found by using the distributive laws to expand. \\[.5em]
    Ordinary distributive law: AND distributes over OR\\
    Special distributive law: OR distributes over AND
    \begin{gather*}
        X(Y+Z)=XY+XZ    \tag{ordinary}  \\
        (X+Y)(X+Z)=X+YZ \tag{special}
    \end{gather*}
    It is also useful to note that
    \[
        (X+Y)(X'+Z) = XZ+X'Y
    \]

    \smallskip\noindent
    The same theorems can be used to convert a SOP expression into a POS expression;
    just go the other way.

    \smallskip\noindent
    \subsubsection*{Exclusive-OR and Equivalence Operations}
    \addcontentsline{toc}{subsection}{Exclusive-OR and Equivalence Operations}
    \vspace{-1em}
    \rule{\linewidth}{.1mm}

    \smallskip\noindent
    Exclusive OR ($\oplus$) can be expressed in familiar terms:
    \[
        X \oplus Y = X'Y + XY'
    \]

    \noindent
    That is, $X\oplus Y=1$  iff one of the terms is true, but not both.
    Hence, ``exclusive.''
    Exclusive OR is commutative, associative, and distributive.

    \smallskip\noindent
    The equivalence operation $(\equiv)$ is the complement of exclusive OR\@.
    \[
        (X\equiv Y) = XY+X'Y'
    \]

    \noindent
    Equivalence is commutative and associative.

    \smallskip\noindent
    In manipulating expressions that contain these operations, it is usually helpful to first rewrite them in terms of AND, OR using the definitions above.
    It is also useful to note that
    \[
        (X'Y+XY')'=XY+X'Y'
    \]

    \smallskip\noindent
    \subsubsection*{The Consensus Theorem}
    \addcontentsline{toc}{subsection}{The Consensus Theorem}
    \vspace{-1em}
    \rule{\linewidth}{.1mm}

    \smallskip\noindent
    Given the terms $X'Y$ and $XZ$, the consensus term is $YZ$.
    The consensus term is the product of all literals from the two factors, excluding the literal whose compliment is in the opposite term.
    For example, the consensus of $A'BC$ and $ACD$ is $BCD$.
    The consensus of $AB'CD$ and $BFGH'$ is $ACDFGH'$.
    The significance of the consensus term is that it is redundant -- it does not change the behaviour of the expression, so we can remove it.
    The consensus theorem and the dual form of the consensus theorem:
    \begin{gather*}
        XY+X'Z+YZ=XY+X'Z                \\
        (X+Y)(X'+Z)(Y+Z)=(X+Y) (X'+Z)
    \end{gather*}

    \begin{center}
        \subsection*{Applying Boolean Algebra}
        \addcontentsline{toc}{section}{Applying Boolean Algebra}
    \end{center}

    \medskip
    \subsubsection*{Steps to Design Logic Circuit}
    \addcontentsline{toc}{subsection}{Steps to Design Logic Circuit}
    \vspace{-1em}
    \rule{\linewidth}{0.1mm}

    \begin{enumerate}
        \item Set up a truth table.
        \item Derive Boolean equations from the truth table.
        \item Simplify the Boolean equation.
        \item Draw a circuit from the Boolean equation.
    \end{enumerate}

    \smallskip
    \subsubsection*{Minterm and Maxterm Expansions}
    \addcontentsline{toc}{subsection}{Minterm and Maxterm Expansions}
    \vspace{-1em}
    \rule{\linewidth}{0.1mm}

    \smallskip\noindent
    A minterm is the product of $n$ literals in which each of the literals appears only once.
    A maxterm is the sum of $n$ literals in which each of the literals appears only once.
    When writing minterms primes correspond to 0's; the opposite applies to maxterms.
    Following DeMorgan's Laws, a minterm and a maxterm are each other's compliments.

    \begin{center}
        \begin{tabular}{c c c}
            $A B C$  & Minterms          & Maxterms                     \\
            0 0 0 0  & $A'B'C'D' = m_0$  & $A + B + C + D = M_0$        \\
            0 0 0 1  & $A'B'C'D = m_1$   & $A + B + C + D' = M_1$       \\
            0 0 1 0  & $A'B'CD' = m_2$   & $A + B + C' + D = M_2$       \\
            $\vdots$ & $\vdots$          & $\vdots$                     \\
            1 1 0 1  & $ABC'D = m{13}$   & $A' + B' + C + D' = M_{13}$  \\
            1 1 1 0  & $ABCD' = m_{14} $ & $A' + B' + C' + D = M_{14}$  \\
            1 1 1 1  & $ABCD = m_{15} $  & $A' + B' + C' + D' = M_{15}$
        \end{tabular}
    \end{center}

    \smallskip\noindent
    A function $f$, defined below, can be expressed in disjunctive normal form, i.e., as a minterm expansion.
    Note that, because a minterm and maxterm are each other's compliment, the equivalent conjunctive normal form, i.e., maxterm expansion, contains only the terms absent from the minterm expansion, and vice versa.
    \begin{gather*}
        f= A'B'C'D' + A'B'C'D + A'BC'D' + A'BCD + AB'C'D' + AB'C'D + ABCD'      \\[.5em]
        f=\Sigma\,m(0,1,4,7,8,9,14) = m_0+m_1+m_4+m_7+m_8+m_9+m_{14}    \\[.5em]
        f=\Pi\,M(2,3,5,6,10,11,12,13,15)=M_2 M_3 M_5 M_6 M_{10}M_{11}M_{12}M_{13}M_{15}
    \end{gather*}

    \pagebreak
    \begin{center}
        \subsection*{Truth Tables}
        \addcontentsline{toc}{section}{Truth Tables}
    \end{center}
    \vspace{4em}
    \begin{multicols}{2}

        \begin{center}

            \subsubsection*{NAND}
            \vspace{1em}
            \begin{tabular}{cc|c}
                X \hspace{1em} & Y \hspace{1em} & \hspace{1em}Z \\
                \hline
                0 \hspace{1em} & 0 \hspace{1em} & \hspace{1em}1 \\
                0 \hspace{1em} & 1 \hspace{1em} & \hspace{1em}1 \\
                1 \hspace{1em} & 0 \hspace{1em} & \hspace{1em}1 \\
                1 \hspace{1em} & 1 \hspace{1em} & \hspace{1em}0 \\
            \end{tabular}

            \vspace{2em}
            $Z = (XY)'$

            \vspace{4em}
            \subsubsection*{XOR}
            \vspace{1em}
            \begin{tabular}{cc|c}
                X \hspace{1em} & Y \hspace{1em} & \hspace{1em}Z \\
                \hline
                0 \hspace{1em} & 0 \hspace{1em} & \hspace{1em}0 \\
                0 \hspace{1em} & 1 \hspace{1em} & \hspace{1em}1 \\
                1 \hspace{1em} & 0 \hspace{1em} & \hspace{1em}1 \\
                1 \hspace{1em} & 1 \hspace{1em} & \hspace{1em}0 \\
            \end{tabular}

            \vspace{2em}
            $Z=X'Y+XY'$
        \end{center}

        \columnbreak
        \begin{center}
            \subsubsection*{NOR}
            \vspace{1em}
            \begin{tabular}{cc|c}
                X \hspace{1em} & Y \hspace{1em} & \hspace{1em}Z \\
                \hline
                0 \hspace{1em} & 0 \hspace{1em} & \hspace{1em}1 \\
                0 \hspace{1em} & 1 \hspace{1em} & \hspace{1em}0 \\
                1 \hspace{1em} & 0 \hspace{1em} & \hspace{1em}0 \\
                1 \hspace{1em} & 1 \hspace{1em} & \hspace{1em}0 \\
            \end{tabular}

            \vspace{2em}
            $Z=(X+Y)'$

            \vspace{4em}
            \subsubsection*{XNOR}
            \vspace{1em}
            \begin{tabular}{cc|c}
                X \hspace{1em} & Y \hspace{1em} & \hspace{1em}Z \\
                \hline
                0 \hspace{1em} & 0 \hspace{1em} & \hspace{1em}1 \\
                0 \hspace{1em} & 1 \hspace{1em} & \hspace{1em}0 \\
                1 \hspace{1em} & 0 \hspace{1em} & \hspace{1em}0 \\
                1 \hspace{1em} & 1 \hspace{1em} & \hspace{1em}1 \\
            \end{tabular}

            \vspace{2em}
            $Z=XY+X'Y'$
        \end{center}
    \end{multicols}

\end{document}
