\documentclass{article}
\usepackage{graphicx}
\usepackage{amsmath}
\usepackage{amsfonts}
\usepackage{esvect}
\usepackage{hyperref}

\begin{document}

    \pagenumbering{roman}
    \tableofcontents
    \pagebreak

    \pagenumbering{arabic}
    \begin{center}
        \subsection*{Motion}
        \addcontentsline{toc}{section}{Motion}
    \end{center}

    \medskip
    \subsubsection*{Projectile Motion}
    \addcontentsline{toc}{subsection}{Projectile Motion}
    \vspace{-1em}
    \rule{\linewidth}{.1mm}

    \smallskip\noindent
    The equations on the left are the standard kinematic equations; they express motion of constant acceleration on a straight line.
    The equations on the right are adapted to express motion on a parabola.
    Projectile motion can be defined as follows:
    An object is launched with initial velocity $\vv{v}_{0}$.
    Horizontal acceleration is zero, and vertical acceleration is $-g$, the downward force of gravity.
    $\theta_0$ is the initial launch angle with respect to the positive $x$-axis.

    \vspace{-1.4em}
    \begin{minipage}[t]{0.4\linewidth}
        \begin{align*}
            &v=v_0+at \\
            &x=x_0+v_0 t+\tfrac{1}{2}at^2 \\
            &v^2=v_0^2+2a\left(x-x_0\right) \\
            &x=x_0+\tfrac{1}{2}\left(v_0+v\right)t \\
            &x=x_0+v t-\tfrac{1}{2}at^2
        \end{align*}
    \end{minipage}
    \begin{minipage}[t]{0.5\linewidth}
        \begin{align*}
            &x=x_0+\left(v_0\cos\theta_0\right)t \\
            &y=y_0+\left(v_0\sin\theta_0\right)t-\tfrac{1}{2}gt^2 \\
            &v_y=v_0\sin\theta_0-gt \\
            &v_y^2=\left(v_0\sin\theta_0\right)^2-2g\left(y-y_0\right)
        \end{align*}
    \end{minipage}

    \vspace{1.2em}
    \smallskip\noindent
    The path trajectory of a projectile is parabolic and can be expressed as
    \[
        y=\left(\tan\theta_0\right)x-\frac{gx^2}{2\left(v_0\cos\theta_0\right)^2}
    \]

    \smallskip\noindent
    If $x_0 = y_0 = 0$, the object's horizontal range $R$ can be expressed as
    \[
        R = \frac{v_0^2}{g}\sin2\theta_0
    \]

    \smallskip
    \subsubsection*{Relative Motion}
    \addcontentsline{toc}{subsection}{Relative Motion}
    \vspace{-1em}
    \rule{\linewidth}{.1mm}

    \smallskip\noindent
    A frame of reference is an arbitrary set of coordinate axes about which we make measurements.
    A passenger on a train is not moving relative to their seat, but they may be moving very quickly from the perspective of a tree next to the train tracks.
    Given two frames of reference, A and B, with constant velocity; the velocity of a particle P with respect to A is equal to the velocity P with respect to B, plus the velocity of B with respect to A .
    \@.   \[
              \vv{v}_{\text{PA}} = \vv{v}_{\text{PB}} + \vv{v}_{\text{BA}}
    \]

    \smallskip\noindent
    The acceleration is equal from either frame of reference.
    \[
        \vv{a}_{\text{PA}} = \vv{a}_{\text{PB}}
    \]

    \begin{center}
        \subsection*{Force and Motion}
        \addcontentsline{toc}{section}{Force and Motion}
    \end{center}

    \medskip
    \subsubsection*{Newton's Laws}
    \addcontentsline{toc}{subsection}{Newton's Laws}
    \vspace{-1em}
    \rule{\linewidth}{.1mm}

    \medskip
    \renewcommand{\arraystretch}{1.3}
    \begin{tabular}{p{0.15\linewidth} p{0.4\linewidth} p{.3\linewidth}}
        First Law:
        & If there is no net force a body, it cannot change its velocity.
        & $\vv{F}_{\text{net}} = 0 \rightarrow \vv{a}=0$ \\
        Second Law:
        & The net force $\vv{F}_{\text{net}}$ on a body is the product of its mass $m$ and its acceleration $\vv{a}$.
        &$\vv{F}_{\text{net}} = m\vv{a}$\\
        Third Law:
        & Equal and opposite reactions.
        &$\vv{F}_{\text{BC}} = - \vv{F}_{\text{CB}}$
    \end{tabular}

    \smallskip
    \subsubsection*{Friction}
    \addcontentsline{toc}{subsection}{Friction}
    \vspace{-1em}
    \rule{\linewidth}{.1mm}

    \smallskip\noindent
    The force of static friction $\vv{f}_s$ has a maximum value $f_{s\text{, max}}$ given by
    \[
        f_{s\text{, max}} = \mu_s F_N
    \]

    \smallskip\noindent
    where $\mu_s$ is the coefficient of static friction and $F_N$ is the magnitude of the normal force.
    If the component of $\vv{F}$ parallel to the surface exceeds $f_{s\text{, max}}$, the body will begin to slide.
    Once the body begins to slide on the surface, the frictional force decreases to a constant value $f_k$ given by
    \[
        f_k=\mu_k F_N
    \]

    \smallskip\noindent
    where $\mu_k$ is the coefficient of kinetic friction.
    Kinetic friction is less forceful than static friction.

    \smallskip
    \subsubsection*{Drag Force}
    \addcontentsline{toc}{subsection}{Drag Force}
    \vspace{-1em}
    \rule{\linewidth}{.1mm}

    \smallskip\noindent
    A body that is moving relative to a surrounding fluid experiences a drag force $\vv{D}$.
    The force points in the direction in which the fluid flows relative to the body, i.e., opposite of the body's motion relative to the fluid.
    The magnitude of $\vv{D}$ can be expressed as
    \[
        D=\tfrac{1}{2}C\rho Av^2
    \]

    \vspace{.2em}
    \smallskip\noindent
    where $C$ is the drag coefficient, $\rho$ is the fluid density, and $A$ is the effective cross-sectional area of the body, and $v$ is the relative speed.

    \smallskip
    \subsubsection*{Terminal Speed}
    \addcontentsline{toc}{subsection}{Terminal Speed}
    \vspace{-1em}
    \rule{\linewidth}{.1mm}

    \smallskip\noindent
    As an object in free fall builds speed, there comes a point where the magnitude of the drag force ${D}$ becomes equal to that of the force of gravity $F_g$.
    The object cannot fall any faster and is said to have reached terminal velocity $v_t$ expressed as
    \[
        v_t = \sqrt{\frac{2F_g}{C\rho A}}
    \]

    \smallskip
    \subsubsection*{Uniform Circular Motion}
    \addcontentsline{toc}{subsection}{Uniform Circular Motion}
    \vspace{-1em}
    \rule{\linewidth}{.1mm}

    \smallskip\noindent
    An object traveling along a circle of radius $R$ at constant speed $v$ is said to be in uniform circular motion.
    Speed is constant, therefore acceleration is constant.
    The constant magnitude of acceleration $\vv{a}$ can be expressed as
    \[
        a=\frac{v^2}{R} \quad \text{and} \quad F_c=\frac{mv^2}{R}
    \]

    \smallskip\noindent
    Where $F_c$ is the magnitude of the centripetal force.
    Acceleration $\vv{a}$ and force $\vv{F_c}$ are always directed toward the center of the circle.
    This is what it means for a force to be centripetal; it is directed toward a center around which an object is moving.

    \medskip\noindent
    The time for the object to complete the circle is
    \[
        T = \frac{2\pi R}{v}
    \]

    \pagebreak
    \begin{center}
        \subsection*{Energy and Work}
        \addcontentsline{toc}{section}{Energy and Work}
    \end{center}

    \smallskip
    \subsubsection*{Kinetic Energy and Work}
    \addcontentsline{toc}{subsection}{Kinetic Energy and Work}
    \vspace{-1em}
    \rule{\linewidth}{.1mm}

    \smallskip\noindent
    The work done on an object over displacement $d$ can be expressed as
    \[
        W=Fd\cos\phi=\vv{F}\cdot\vv{d}
    \]
    \noindent
    where $\phi$ is the angle between $\vv{F}$ and $\vv{d}$, and both $\vv{F}$ and $\phi$ are constant.

    \medskip\noindent
    Following the law of conservation of energy, the kinetic energy of an object is equal to the energy applied to that object in order to accelerate it to a given velocity.
    That concept is captured by the work-kinetic energy theorem: an increase in kinetic energy means that energy is being transferred to the object, and a decrease in kinetic energy means that energy is being transferred from the object; positive work and negative work, respectively.
    \begin{gather*}
        K=\frac{1}{2}mv^2                   \\
        \Delta K = K_f - K_i = W            \\[.3em]
        K_f = K_i + W
    \end{gather*}

    \smallskip
    \subsubsection*{Gravity and Work}
    \addcontentsline{toc}{subsection}{Gravity and Work}
    \vspace{-1em}
    \rule{\linewidth}{.1mm}

    \smallskip\noindent
    The work done on an object by the force of gravity can be expressed as
    \[
        W_g = mgd\cos\phi
    \]
    \noindent
    where $\phi$ is the angle between $\vv{F}_g$ and $\vv{d}$, and $\phi$ is constant.

    \medskip\noindent
    If $K_f=K_i$\,, then by the work-kinetic energy theorem, the work applied in lifting an object can be expressed as
    \[
        W_a = -W_g
    \]


\end{document}
